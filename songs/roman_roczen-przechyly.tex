\newpage
\begin{song}{title={Przechyły}, music={Roman Roczeń}}
    \begin{verse}
        Pierwszy ^{a}raz, przy ^{h}pełnym takie^{e}lunku \\
        Biorę s^{a}ter i ^{h}trzymam kurs na ^{e}wiatr \\
        I jest ^{a}jak przy ^{h}pierwszym poca^{e}łunku \\
        W ustach ^{C}sól, go^{H7}rącej wody ^{e}smak
    \end{verse}
    \begin{chorus}
        O ho ^{a}ho, prze^{h}chyły i prze^{e}chyły \\
        O ho ^{a}ho, za ^{h}falą fala m^{e}knie \\
        O ho ^{a}ho, trzy^{h}majcie się dziew^{e}czyny (za liny!) \\
        Ale ^{C}wiatr, ó^{H7}semka chyba ^{e}dmie $\times 2$
    \end{chorus}
    \begin{verse}
        Zwrot przez sztag? Okej, zaraz zrobię \\
        Słyszę, jak kapitan cicho klnie \\
        Gubię wiatr i zamiast w niego dziobem \\
        To on mnie od tyłu, kumple w śmiech
    \end{verse}
    \begin{chorus}
        O ho ho, przechyły i przechyły\ldots
    \end{chorus}
    \begin{verse}
        Ej, ty tam, za burtę wychylony \\
        Tu naprawdę się nie ma z czego śmiać \\
        Cicho siedź i lepiej proś Neptuna \\
        Żeby coś nie spadło ci na kark
    \end{verse}
    \begin{chorus}
        O ho ho, przechyły i przechyły\ldots
    \end{chorus}
    \begin{verse}
        Krople mgłyi, w różowych kropel pyle \\
        Tańczy jacht, po deskach spływa cień \\
        Jutro znów wypłynę, bo odkryłem \\
        Że wciąż brzmi żeglarska, stara pieśń
    \end{verse}
    \begin{chorus}
        O ho ho, przechyły i przechyły\ldots
    \end{chorus}
\end{song}

