\newpage
\begin{song}{title={W domach z betonu nie ma wolnej miłości}, music={Martyna Jakubowicz}, capo=3}
    \begin{intro}
        \writechord{e} \writechord{G} \writechord{h} \writechord{A} \\\
        \writechord{e} \writechord{e} \writechord{C} \writechord{D}
    \end{intro}
    \begin{verse}
        ^{e}Obudziłam się ^{G}później niż zwykle \\
        ^{h}Wstałam z łóżka, w radiu b^{A}yła muzyka -^{e}-- --- -^{C}-- -^{D}-- \\
        ^{e}Najpierw zdjęłam kosz^{G}ule potem trochę ta^{h}ńczyłam \\
        I przez chwile się cz^{A}ułam jak dziewczyny w świerszc^{e}zykach --- -^{C}-- -^{D}-- \\
    \end{verse}
    \begin{chorus}
        ^{C}W domach z bet^{G}onu \\
        -^{h}-- ^{A}Nie ma wolnej miło^{e}ści \\
        ^{C}Są stosunki małże^{G}ńskie oraz akty nierz^{h}ądne \\
        Casanova tu ^{A}u nas nie gości \\
    \end{chorus}
    \begin{verse}
        Ten z przeciwka co ma kota i rower \\
        Stał przy oknie nieruchomo jak skała \\
        Pomyślałam "to dla ciebie ta rewia \\
        Rusz się, przecież nie będę tak stała" \\
    \end{verse}
    \begin{chorus}
        W domach z betonu \ldots \\
    \end{chorus}
    \begin{verse}
        Po południu zobaczyłam go w sklepie \\
        Patrzył we mnie jak w jakiś obrazek \\
        Ruchem głowy pokazał mi okno \\
        Wiec ten wieczór spędzimy znów razem \\
    \end{verse}
    \begin{chorus}
        W domach z betonu \ldots \\
    \end{chorus}
\end{song}

