\newpage
\begin{song}{title={Piła tango}, music={Strachy na Lachy}}
    \normalsize
	\begin{intro}
	    \writechord{a} \writechord{a} \writechord{d} \writechord{E} $\times 3$ \\
        \writechord{a} \writechord{a} \writechord{d}
	\end{intro}
    \begin{verse*}
        Oto ^{E}historia z ^{a}kantem ^{d} \\
        Co pod^{E}wójne ma dn^{a}o ^{d} \\
        Gdyby na^{E}pisał ją ^{a}Dante ^{d} \\
        To nie ^{E}tak by to szł^*{a}o \ldots ^{d} ^{E}
    \end{verse*}
    \begin{verse}
        Grzesiek ^{a}Kubiak, czyli ``Kuba'', rządził ^{d}naszą podsta^{E}wówką \\
        Po ^{a}lekcjach na boisku ganiał ^{d}za mną z cegł^{E}ówką \\
        W Pile było jak w Chile, każdy miał czerwone ryło \\
        Mniej lub bardziej to pamiętasz --- spytaj jak to było \\
        W czasach, gdy nad Piłą jeszcze latały samoloty \\
        Wojewoda Śliwiński kazał pomalować płoty \\
        Potem wszystkie płoty w Pile miały kolor zieleni \\
        Rogaczem na wieżowcu Piła witała jeleni
    \end{verse}
    \begin{verse*}
        \writechord{E7}
    \end{verse*}
    \begin{chorus}
        Statek ^{a}Piła Tango ^{d7} ^{E7} \\
        Czar^{a}na bandera ^{d7} ^{E7} \\
        To tylko Piła tango \\
        Tańczysz to teraz \\
        Płynie statek Piła Tango \\
        Czarna bandera \\
        Ukłoń się świrom \\
        Żyj, nie umieraj
    \end{chorus}
    \newpage
    \begin{verse}
        Gruby jak armata Szczepan błąkał się po kuli ziemskiej \\
        Trafił do Ameryki prosto z Legii Cudzoziemskiej \\
        Baca w Londynie z Buchami się sąsiedzi \\
        Lżej się tam halucynuje, nikt go tam nie śledzi \\
        Karawan z Holandią przyjechał tutaj wreszcie \\
        Są już Kula, Czarny Dusioł --- słychać strzały na mieście \\
        Znam jednak takie miejsca, gdzie jest lepiej chodzić z nożem \\
        Całe Górne i Podlasie, wszyscy są za Kolejorzem (hej, Kolejorz!)
    \end{verse}
    \begin{chorus}
        Statek Piła Tango\ldots
    \end{chorus}
    \begin{verse}
        Andrzej Kozak, Mandaryn --- znana postać medialna \\
        Tyci przy nim jest kosmos, gaśnie Gwiazda Polarna \\
        Jest tu Siwy, który w rękach niebezpieczne ma narzędzie \\
        A kiedy Siwy tańczy, znaczy mordobicie będzie \\
        U Budzików Pod Tytułem chleją nawet z gór szkieły \\
        Zbigu śpi przy stoliku, ma nieczynny przełyk \\
        Lecz spokojnie, panowie, według mej najlepszej wiedzy \\
        Najszersze gardła tu to mają z INRI koledzy
    \end{verse}
    \begin{chorus}
        Statek Piła Tango\ldots
    \end{chorus}
    \begin{verse}
        Nad rzeką, latem ferajna na grilla się zasadza \\
        Auta z Niemiec? Sam wiem, kto je tu sprowadza \\
        Żaden spleen i cud, na ulicach nie śpią złotówki \\
        W Pile Święta jest Rodzina\footnotemark{} i święte są żarówki \\
        Nic nie szkodzi, że z wieczora miasto dławi się w fetorach \\
        Ważne, że jest żużel i kiełbasy senatora \\
        Fajne z Wincentego Pola idą w świat dziewczyny \\
        Po pokładzie jeździ Jojo bicyklem z Ukrainy
        \footnotetext{%
            Parafia pw. Świętej Rodziny w Pile
        }
    \end{verse}
    \begin{chorus}
        Statek Piła Tango\ldots
    \end{chorus}
    \begin{interlude}
        Oto historia z kantem \\
        Co podwójne ma dno \\
        Gdyby napisał ją Dante \\
        To nie tak by to szło\ldots \\
        (by szło, by szło\ldots)
    \end{interlude}
\end{song}

