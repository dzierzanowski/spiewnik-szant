\newpage
\begin{song}{title={Shenandoah}, music={tradycyjna}}
    \begin{verse}
        ^{D}O Missouri, Ty wielka rz^{G}e --- k^{D}o \\
        Ojcze rz^{G}ek, kto bieg twój zmie^{D}rzy \\
        Wigwamy I^{h}ndian na jej b^{D}rzegu \\
        A^{D}way, gdy czółno m^{f#}knie -^{G}-- \\
        Poprzez nu^{D}rt Misso^{G}u --- r^{D}i
    \end{verse}    
    \begin{verse}
        O Shenandoah, jej imię było \\
        Ojcze rzek, kto bieg twój zmierzy \\
        I nie wiedziała, co to miłość \\
        Away, gdy czółno mknie  \\
        Poprzez nurt Missouri
    \end{verse}    
    \begin{verse}
        Aż przybył kupiec i w rozterce \\
        Ojcze rzek, kto bieg twój zmierzy \\
        Jej własne ofiarował serce \\
        Away, gdy czółno mknie  \\
        Poprzez nurt Missouri
    \end{verse}   
     \begin{verse}
        Lecz stary wódz rzekł, że nie może \\
        Ojcze rzek, kto bieg twój zmierzy \\
        Białemu córka wodza ścielić łoże \\
        Away, gdy czółno mknie  \\
        Poprzez nurt Missouri
    \end{verse}   
    \begin{verse}
        Lecz wódka białych wzrok mu mami \\
        Ojcze rzek, kto bieg twój zmierzy \\
        Już wojownicy śpią z duchami \\
        Away, gdy czółno mknie  \\
        Poprzez nurt Missouri
    \end{verse}    
    \begin{verse}
        Wziął czółno swe i z biegiem rzeki \\
        Ojcze rzek, kto bieg twój zmierzy \\
        Dziewczynę uwiózł w kraj daleki \\
        Away, gdy czółno mknie  \\
        Poprzez nurt Missouri
    \end{verse}    
    \begin{verse}
        O, Shenandoah, czerwony ptaku \\
        Ojcze rzek, kto bieg twój zmierzy \\
        Wraz ze mną płyń po życia szlaku \\
        Away, gdy czółno mknie  \\
        Poprzez nurt Missouri
    \end{verse}
\end{song}

