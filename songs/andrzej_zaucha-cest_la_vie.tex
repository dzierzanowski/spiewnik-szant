\newpage
\begin{song}{title={C'est la vie (Paryż z pocztówki)}, music={Wiesław Pieregorólka}, lyrics={Jacek Cygan}, interpret={Andrzej Zaucha}}
    \begin{multicols}{2}
    \begin{intro}
    \writechord{B} \writechord{F/A} \writechord{g7} \writechord{C} \writechord{F} \writechord{F7/A} \\ 
    \writechord{B} \writechord{F/A} \writechord{g7} \writechord{g7/C} \\
    \writechord{F} \writechord{B/F} \writechord{F} \writechord{F7/A}
    \end{intro}
    \begin{verse}
        ^{B}C'est la v^{F/A}ie \\
        Cały Twój P^{g7}aryż z pocz^{C}tówek i ^{F}mgły --^{F7}- \\
        ^{B}C'est la v^{F/A}ie \\
        Wymyślon^{g7}y --^{g7/C}-  \\
        ^{B}C'est la v^{F/A}ie \\
        Ciebie obch^{g7}odzi, przejmu^{g7/C}jesz się ty^{F}m \\
    \end{verse}
    \begin{intro}
        \textit{play intro again}
    \end{intro}
    \begin{verse}
        ^{B}C'est la v^{F/A}ie \\
         Podmiejski po^{g7}ciąg rozw^{C}ozi Twe d^{F}ni --^{F7}- \\
        ^{B}C'est la v^{F/A}ie \\
         Wciąż spóź^{g7}niony --^{g7/C}-  \\
        ^{B}C'est la v^{F/A}ie \\
         Cały Twój P^{g7}aryż, dwie dr^{C}ogi na kr^{F}zyż --^{F7}- \\ 

        ^{B}Knajpa, ko^{F/A}ściół -^{g7}-- wi^{g7/C}dok z mo^{F}stu --^{F7} \\
        ^{B}Knajpa, ko^{F/A}ściół i ten b^{g7}ruk - i^{g7/C}deał nie tknął g^{F}o \\ 
    \end{verse}
    \begin{interlude}
        \writechord{F} \writechord{F/A} \\
        \writechord{B} \writechord{B/G} \writechord{B/C} $\times 2$
    \end{interlude}
    \begin{verse}
        ^{B}Knajpa, ko^{F/A}ściół -^{g7}-- wi^{g7/C}dok z mo^{F}stu --^{F7} \\
        ^{B}Knajpa, ko^{F/A}ściół i ten b^{g7}ruk - tak re^{g7/C}alny \\ 
    \end{verse}
    \begin{chorus}
        ^{C#}Zostaniesz t^{G#/C}u - ile m^{b7}ożna tak ż^{Eb}yć na pa^{G#}lcach -^{G#7}-- \\
        ^{C#}Zostaniesz t^{G#/C}u - po złud^{b7}zenia -^{b7/Eb}-- \\
        ^{H}Zostaniesz t^{F#}u - w "Kaskadzie" n^{g#7}ocą też gr^{C#7}ają wa^{F#}lca \\
        ^{H}Już na rogu ku^{F#}mple, jak grzech \\
        -^{g#}-- Odwr^{C#}otu już ni^{F#}e ma \\
        Nie, n^{F#7}ie nie \\
        ^{H}Wypijesz to wsz^{F#}ystko do dna, także dz^{g#7}iś \\
        ^{g#7/c#}Jak c^{H}o dnia \\
    \end{chorus}
    \begin{interlude}
        \writechord{H} \writechord{F#/B} \writechord{g#7} \writechord{C#} \writechord{F#} \writechord{F#7/B} \\ 
        \writechord{H} \writechord{F#/B} \writechord{g#7} \writechord{g#7/C#} \\
        \writechord{F#} \writechord{H/F#} \writechord{F#} \writechord{F#7/B}
    \end{interlude}
    \begin{verse}
        \textit{(akordy jak na początku tylko pół tonu wyżej)}
        C'est la vie \\
        Cały Twój Paryż z pocztówek i mgły \\
        C'est la vie \\
        Wymyślony  \\
        C'est la vie \\ 
        Cały Twój Paryż, dwie drogi na krzyż \\ 
        Knajpa, kościół - widok z mostu \\
        Knajpa, kościół i ten bruk - ideał nie tknął go \\
    \end{verse}
    \end{multicols}
\end{song}

