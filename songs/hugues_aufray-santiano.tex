\newpage
\newcommand{\tqs}{\textquotesingle}
\begin{song}{title={Santiano}, interpret={Hugues Aufray}, music={tradycyjna}, capo=2}
    \begin{verse}
        C\tqs est ^{e}un fameux trois-mâts, fin comme un oise^{D}au \\
        Hissez ^{e}haut ! « Santi^{D}ano » ! \\
        ^{a}Dix-huit nœuds, quatre ^{D}cents tonne^{h}aux \\
        Je suis ^{e}fier d\tqs y ^{h}être ^{e}matelot
    \end{verse}
    \begin{chorus}
        Tiens ^{e}bon la vague et tiens bon le ^{D}vent \\
        Hissez ^{e}haut ! « Santi^{D}ano » ! \\
        ^{a}Si Dieu veut, toujours ^{D}droit de^{h}vant \\
        Nous i^{e}rons jus^{h}qu\tqs à San ^{e}Francisco
    \end{chorus}
    \begin{verse}
        Je pars pour de longs mois en laissant Margot \\
        Hissez haut ! « Santiano » ! \\
        D\tqs y penser, j\tqs avais le cœur gros \\
        En doublant les feux de Saint-Malo
    \end{verse}
    \begin{chorus}
        Tiens bon la vague et tiens bon le vent\ldots
    \end{chorus}
    \begin{verse}
        On prétend que là-bas, l\tqs argent coule à flots \\
        Hissez haut ! « Santiano » ! \\
        On trouve l\tqs or au fond des ruisseaux \\
        J'en ramènerai plusieurs lingots
    \end{verse}
    \begin{chorus}
        Tiens bon la vague et tiens bon le vent\ldots
    \end{chorus}
    \begin{verse}
        Un jour je reviendrai, chargé de cadeaux \\
        Hissez haut ! « Santiano » ! \\
        Au pays, j\tqs irai voir Margot \\
        À son doigt, je passerai l\tqs anneau
    \end{verse}
    \begin{chorus}
        Tiens bon le cap et tiens bon le flot \\
        Hissez haut ! « Santiano » ! \\
        Sur la mer qui fait le gros dos \\
        Nous irons jusqu\tqs à San Francisco $\times 2$
    \end{chorus}
\end{song}

